\documentclass[]{imag-ms-template}

\title{Author Usage Template for MIT Journals} 

\author{Author 1,$^{1\ast}$ Author 2,$^{1}$ Author 3$^{2}$\\
{\small $^{1}$Department of Chemistry, University of Wherever,}\\
{\small An Unknown Address, Wherever, ST 00000, USA}\\
{\small $^{2}$Another Unknown Address, Palookaville, ST 99999, USA}\\
{\small $^\ast$Correspondence:  jsmith@wherever.edu}
}

\addbibresource{sample.bib}

\begin{document} 

\maketitle 


\keywords{keywords1, keywords2 and keywords3}

\begin{abstract}
  This document presents a number of hints about how to set up your
  paper in \LaTeX.  We provide a template file,
  \texttt{imag-ms-template-instr.tex}, that you can use to set up the
  \LaTeX\ source for your article.  An example of the style is the
  special \texttt{\{abstract\}} environment used to set up the
  abstract you see here.
\end{abstract}

\section{Introduction}

In this file, we present some tips and sample mark-up to ensure that your
\LaTeX\ file has the smoothest possible journey from review manuscript
to published paper.  We focus here particularly on
issues related to headings, citations, math, tables, and
figures, as those tend to be the biggest sticking points.  Please use
the source file for this document, \texttt{imag-ms-template-instr.tex}, as a template
for your manuscript, cutting and pasting your content into the file at
the appropriate places.

% now two fake paragraphs to illustrate how paragraph
% spacing/indentation looks

%Lorem ipsum dolor sit amet, consectetur adipiscing elit, sed do
%eiusmod tempor incididunt ut labore et dolore magna aliqua. Ut enim ad
%minim veniam, quis nostrud exercitation ullamco laboris nisi ut
%aliquip ex ea commodo consequat. Duis aute irure dolor in
%reprehenderit in voluptate velit esse cillum dolore eu fugiat nulla
%pariatur. Excepteur sint occaecat cupidatat non proident, sunt in
%culpa qui officia deserunt mollit anim id est laborum.

%Lorem ipsum dolor sit amet, consectetur adipiscing elit, sed do
%eiusmod tempor incididunt ut labore et dolore magna aliqua. Ut enim ad
%minim veniam, quis nostrud exercitation ullamco laboris nisi ut
%aliquip ex ea commodo consequat. Duis aute irure dolor in
%reprehenderit in voluptate velit esse cillum dolore eu fugiat nulla
%pariatur. Excepteur sint occaecat cupidatat non proident, sunt in
%culpa qui officia deserunt mollit anim id est laborum.

\newpage

\subsection{Front Matter}

Please use the below tags for the article front matter:

\begin{verbatim}
\title{Article Title} 

\author{Author 1,$^{1\ast}$ Author 2,$^{1}$ Author 3$^{2}$\\
{\small $^{1}$Department of Chemistry, University of Wherever,}\\
{\small An Unknown Address, Wherever, ST 00000, USA}\\
{\small $^{2}$Another Unknown Address, Palookaville, ST 99999, USA}\\
{\small $^\ast$Correspondence:  jsmith@wherever.edu}
}

\maketitle
\end{verbatim}

\subsection{Abstract}

Use the tag:

\begin{verbatim}
\begin{abstract}
  This document presents a number of hints about how to set up your
  paper in \LaTeX.  We provide a template file,
  \texttt{imag-ms-template-instr.tex}, that you can use to set up the
  \LaTeX\ source for your article.  An example of the style is the special
  \texttt{\{abstract\}} environment used to set up the abstract you
  see here.
\end{abstract}
\end{verbatim}

\subsection{Headings}

Use the standard tags \verb!\section, \subsection, \subsubsection, \paragraph!, and \verb!\subparagraph! for the Headings \verb!H1, H2, H3, H4!, and \verb!H5!, respectively.

\subsection{Handling Math, Tables, and Figures}

We suggest using the \verb!mathtools.sty! file to get various 
display math styles. A few of the codes are given below for easy
reference:

\bigskip

\noindent\begin{tabular}{@{}ll}
\toprule
\verb!equation!\\
\verb!align!\\
\verb!\[...\]! or \verb!equation*!\\
\verb!gather!\\
Various types of matrices, e.g., \verb!pmatrix!, \verb!bmatrix!,
\verb!vmatrix!, \verb!smallmatrix!,\\
\verb!alignat!, etc.\\
\bottomrule
\end{tabular}


\subsection{Tables}  

We suggest using the \verb!threeparttable.sty! file to format the tables
and their notes properly. Examples are given below:

\begin{verbatim}
\begin{table}
\begin{threeparttable}
\caption{Time of the Transition Between Phase 1 and Phase 2\tnote{$a$}
\label{tab:label}}
\begin{tabular}{@{}ll}
\toprule
 Run  & Time (min)  \\
\midrule
  \textit{l}1  & 260   \\
  \textit{l}2  & 300   \\
  \textit{l}3  & 340   \\
  \textit{h}1  & 270   \\
  \textit{h}2  & 250   \\
  \textit{h}3  & 380   \\
  \textit{r}1  & 370   \\
  \textit{r}2  & 390   \\
\bottomrule
\end{tabular}
\begin{tablenotes}[flushleft]\footnotesize
\item[${a}$]Table note text here.
\end{tablenotes}
\end{threeparttable}
\end{table}
\end{verbatim}

\subsubsection*{Output}

\begin{table}[h!]
\begin{threeparttable}
\caption{Time of the Transition Between Phase 1 and Phase 2\tnote{$a$}
\label{tab:label}}
\setlength{\tabcolsep}{45pt}%
\begin{tabular}{@{}ll}
\toprule
 Run  & Time (min)  \\
\midrule
  \textit{l}1  & 260   \\
  \textit{l}2  & 300   \\
  \textit{l}3  & 340   \\
  \textit{h}1  & 270   \\
  \textit{h}2  & 250   \\
  \textit{h}3  & 380   \\
  \textit{r}1  & 370   \\
  \textit{r}2  & 390   \\
\bottomrule
\end{tabular}
\begin{tablenotes}[flushleft]\footnotesize
\item[${a}$]Table note text here.
\end{tablenotes}
\end{threeparttable}
\end{table}

\subsection*{Spanning rules}

Use \verb!\cmidrule! to obtain spanning of rules from column to
column. Usage is

\begin{verbatim}
\cmidrule{fromcolumn-tocolumn}, e.g., \cmidrule{2-3}.
\end{verbatim}

\subsection{Figures}

Figure callouts within the text should be
in the form of \LaTeX\ references; for example, \verb+\ref{fig1}+.

%% For the
%% figures themselves, treatment can differ depending on whether the
%% manuscript is an initial submission or a final revision for acceptance
%% and publication.  For an initial submission and review copy, you can
%% use the \LaTeX\ \verb+{figure}+ environment and the
%% \verb+\includegraphics+ command to include your PostScript figures at
%% the end of the compiled file.  For the final revision,
%% however, the \verb+{figure}+ environment should \textit{not} be used;
%% instead, the figure captions themselves should be typed in as regular
%% text at the end of the source file (an example is included here), and
%% the figures should be uploaded separately according to the Art
%% Department's instructions.

For inclusion of figures (e.g., Fig~\ref{example_figure}), please use code such as:

\begin{verbatim}
\begin{figure}[htbp]\begin{center}\includegraphics[width=0.2\textwidth]{figure}
\caption{Example caption text.}
\label{example_figure}\end{center}\end{figure}
\end{verbatim}

\begin{figure}[htbp]\begin{center}\includegraphics[width=0.2\textwidth]{figure}
\caption{Example caption text.}
\label{example_figure}\end{center}\end{figure}


Please use \verb!\begin{sidewaystable}...\end{sidewaystable}! and\\
\verb!\begin{sidwaysfigure}...\end{sidewaysfigure}! to get rotating figures/tables.


\section{Algorithms}

For \verb@Algorithms@, please use the standard \LaTeX\ supporting file
\verb!algorithm2e.sty!; the format and the output are given below:

\begin{verbatim}
\begin{algorithm}[h!]
\SetAlgoLined
\SetKwFunction{IL}{InitializeDistance}
\SetKwFunction{PL}{PropagateInsertion}
\SetKwFunction{MIN}{Min}
\SetKwFunction{MX}{Max}
\SetKwFunction{TOP}{Top}
\SetKwFunction{Push}{Push}
\SetKwFunction{Pop}{Pop}
\SetKwFunction{Append}{Append}
\SetKwData{Queue}{Queue}
\KwResult{The length of shortest path from $s$ to $t$}
 $PreviousLayer=[s]$\;
 $s.distance = 0$\;
 \For(\tcc*[f]{Do the computation layer by layer}){i = 1 \KwTo m}{
   $CurrentLayer = [(i,v_1),(i,v_{2}),\ldots, (i,v_{n}), (i,k)]$\;
   $x.distance = \infty \ \forall  x \in CurrentLayer$\;
   \IL{PreviousLayer,CurrentLayer}\;
   \PL{CurrentLayer}\;
   $PreviousLayer = CurrentLayer$\;
 }
 \KwRet{\MIN{PreviousLayer.distance}}\;
 \caption{Algorithm for sequence to graph alignment}
 \label{algo:linear}
\end{algorithm}
\end{verbatim}

\subsection*{Output}

\begin{algorithm}[h!]
\SetAlgoLined
\SetKwFunction{IL}{InitializeDistance}
\SetKwFunction{PL}{PropagateInsertion}
\SetKwFunction{MIN}{Min}
\SetKwFunction{MX}{Max}
\SetKwFunction{TOP}{Top}
\SetKwFunction{Push}{Push}
\SetKwFunction{Pop}{Pop}
\SetKwFunction{Append}{Append}
\SetKwData{Queue}{Queue}
\KwResult{The length of shortest path from $s$ to $t$}
 $PreviousLayer=[s]$\;
 $s.distance = 0$\;
 \For(\tcc*[f]{Do the computation layer by layer}){i = 1 \KwTo m}{
   $CurrentLayer = [(i,v_1),(i,v_{2}),\ldots, (i,v_{n}), (i,k)]$\;
   $x.distance = \infty \ \forall  x \in CurrentLayer$\;
   \IL{PreviousLayer,CurrentLayer}\;
   \PL{CurrentLayer}\;
   $PreviousLayer = CurrentLayer$\;
 }
 \KwRet{\MIN{PreviousLayer.distance}}\;
 \caption{Algorithm for sequence to graph alignment}
 \label{algo:linear}
\end{algorithm}

\section{Lists}

Please use the standard tags for Numbered lists and Bulleted lists; for example,

\subsection*{Numbered lists}

\begin{verbatim}
\begin{enumerate}
\item Text for first-level numbered lists text text text text
Text for first-level numbered lists text text text text: 
\begin{enumerate}
\item Text for second level numbered lists text text text text
Text for second level numbered lists text text text text 
\item Text text text text Text for second level numbered lists
text text text text 
\end{enumerate}
\item Text text text text Text for first-level numbered lists
text text text text 
\end{enumerate}
\end{verbatim}

\subsection*{Output}

\begin{enumerate}
\item Text for first-level numbered lists text text text text Text for first-level numbered lists text text text text: 
\begin{enumerate}
\item Text for second level numbered lists text text text text Text for second level numbered lists text text text text 
\item Text text text text Text for second level numbered lists text text text text 
\end{enumerate}
\item Text text text text Text for first-level numbered lists text text text text 
\end{enumerate}

\subsection*{Bulleted lists}

\begin{verbatim}
\begin{itemize}
\item Text for first-level bulleted lists text text text text
Text for first-level bulleted lists text text text text 
\begin{itemize}
\item text for second level bulleted lists text text text text
Text for second level bulleted lists text text text text 
\item text text text text Text for second level bulleted lists
text text text text 
\end{itemize}
\item Text text text text Text for first-level bulleted lists
text text text text 
\end{itemize}
\end{verbatim}

\subsection*{Output}

\begin{itemize}
\item Text for first-level bulleted lists text text text text Text for first-level bulleted lists text text text text 
\begin{itemize}
\item text for second level bulleted lists text text text text Text for second level bulleted lists text text text text 
\item text text text text Text for second level bulleted lists text text text text 
\end{itemize}
\item Text text text text Text for first-level bulleted lists text text text text 
\end{itemize}

\subsection*{Extract/Quote}

Use the standard tag \verb!\begin{quote}...\end{quote}! for quoted text; for example,

\begin{verbatim}
\begin{quote}
Text for quoted text text text text text text text text text text text
text text text text text text text text text text text text text text 
\end{quote}
\end{verbatim}

\subsection*{Output}

\begin{quote}
Text for quoted text text text text text text text text text text text
text text text text text text text text text text text text text text 
\end{quote}

\section{Footnote}

Use the standard \LaTeX\ tag \verb!\footnote! to get footnotes at the
bottom of the page.

\section{Special Fonts}

Use the standard \LaTeX\ tags \verb!\mathcal!, \verb!\mathscr!, and
\verb!\mathbb! to get characters in special fonts such as
$\mathcal{A}, \mathscr{A}$, and $\mathbb{A}$, respectively.

\section{Enunciation or Math Heads}

Generally \verb!theorem!, \verb!lemma!, etc., are called
Enunciation or Math heads. In this template, we define some standard
enunciations (\verb!theorem!, \verb!lemma!, \verb!corollary!).

\subsection*{Sample Input/Output}

\subsection{Input}

\begin{verbatim}
\begin{theorem}
This is test for math head ``Theorem'' text text text text.
\end{theorem}
\end{verbatim}

\subsection{Output}

\begin{theorem}
This is test for math head ``Theorem'' text text text text.
\end{theorem}

\subsection{Define Own Math Heads/Enunciation}

You are allowed to define your own enunciations; the format is given
below:

\begin{verbatim}
\newtheorem{short name of the head}{Head to Display}
\end{verbatim}

\subsection*{Example}

If you need to define a group of text under the head ``Proposition,''
then you have to define it as 

\begin{verbatim}
\newtheorem{proposition}{Proposition}
\end{verbatim}

\newtheorem{proposition}{Proposition}
\begin{proposition}
This is a test for math head ``Proposition'' text text text text
\end{proposition}

\subsection{Unnumbered Math Heads/Enunciation}

Just introduce \verb!*!, which makes the numbered math head text
into an unnumbered math head; for example,

\begin{verbatim}
\begin{theorem*}
This is a test for unnumbered math head ``Theorem'' text text text text
\end{theorem*}
\end{verbatim}

\begin{theorem*}
This is a test for unnumbered math head ``Theorem'' text text text text
\end{theorem*}

\section{Bibliography/References with APA Style}

As per MIT standards, we fixed the Reference style \verb!APA! in
the template with the combination of the supporting file
\verb!biblatex! and \verb!natbib! options, which
help to achieve various types of bibliography cross links. Those details are given below:

\subsection{Formatting Citations}

\noindent\begin{tabular}{@{}ll}
\bf Type&\bf Results\\
\midrule
\verb+\citet{ref2}+&Goossens et al. (1993)\\
\verb+\citet[chap. 2]{ref2}+&Goossens et al. (1993, chap. 2)\\
    \verb+\citep{ref2}+	    &   	(Goossens et al., 1993)\\
    \verb+\citep[chap. 2]{ref2}+ 	&    	(Goossens et al., 1993, chap. 2)\\
    \verb+\citep[see][]{ref2}+ 	 &    	(see Goossens et al., 1993)\\
    \verb+\citep[see][chap. 2]{ref2}+ 	&    	(see Goossens et al., 1993, chap. 2)\\
    \verb+\citet*{ref2}+ 	    &    	Goossens, Mittelbach, and Samarian (1993)\\
    \verb+\citep*{ref2}+	    &    	(Goossens, Mittelbach, \& Samarian
    1993) \\
\end{tabular}

\bigskip

\noindent {\bf Note:} Please use {\bf biber} (biber.exe in Windows) to get
better output for References.

\nocite{*}

\subsection{Example Citations}

See \cite{Einstein1905} and \cite{Goossens1993,Knuth1986}. Also see \cite{Chen2023}.


\section{Note to User}

We have already included almost all the required \texttt{.sty} files in the
\LaTeX\ template \verb!imag-ms-template.cls!; hence, there is no need to call those in
your \texttt{.tex} application files.

%%%% commented out as it doesn't work, but isn't really needed anyway
%\section{To add any instruction to Comp}
%
%Please use the tag \verb!\notetocomp! to display any important
%note/info to typesetter/comp/publisher, this will produce the output
%in margin, example shown below:
%
%\begin{verbatim}
%\notetocomp{Note to comp/publisher}
%\end{verbatim}
%
%\subsection*{Output}
%
%This is for test \notetocomp{Note to comp/publisher} this is for test

\subsection*{General Notes}

\noindent This template will work in most recent \TeX\ distributions
(e.g., MiKTeX, TeXLive) with any type of \TeX\ engines, such as
\LaTeX, PDF\LaTeX, Xe\LaTeX, and Lua\LaTeX, as well as in all types of OS, such
as MS-Windows, Mac OS X, and Linux. It will also work well in Overleaf.

\section*{Data and Code Availability}

Data and Code Availability text (mandatory unless there is no data or code used).

\section*{Author Contributions}

Author Contributions text (mandatory).

\section*{Funding}

Funding text (optional).

\section*{Declaration of Competing Interests}

Declaration of Competing Interests text (mandatory).

\section*{Acknowledgements}

Acknowledgements text (optional).

\section*{Supplementary Material}

Supplementary Material (created during production as a web link to online material).

\printbibliography

\appendix

\section{Appendix}

Appendices (optional).


\end{document}
