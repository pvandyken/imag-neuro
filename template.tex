% Options for packages loaded elsewhere
\PassOptionsToPackage{unicode}{hyperref}
\PassOptionsToPackage{hyphens}{url}
\PassOptionsToPackage{dvipsnames,svgnames,x11names}{xcolor}
%
\documentclass[]{imag-ms-template}

\usepackage{amsmath,amssymb}
\usepackage{iftex}
\ifPDFTeX
  \usepackage[T1]{fontenc}
  \usepackage[utf8]{inputenc}
  \usepackage{textcomp} % provide euro and other symbols
\else % if luatex or xetex
  \usepackage{unicode-math}
  \defaultfontfeatures{Scale=MatchLowercase}
  \defaultfontfeatures[\rmfamily]{Ligatures=TeX,Scale=1}
\fi
\usepackage{lmodern}
\ifPDFTeX\else  
    % xetex/luatex font selection
\fi
% Use upquote if available, for straight quotes in verbatim environments
\IfFileExists{upquote.sty}{\usepackage{upquote}}{}
\IfFileExists{microtype.sty}{% use microtype if available
  \usepackage[]{microtype}
  \UseMicrotypeSet[protrusion]{basicmath} % disable protrusion for tt fonts
}{}
\makeatletter
\@ifundefined{KOMAClassName}{% if non-KOMA class
  \IfFileExists{parskip.sty}{%
    \usepackage{parskip}
  }{% else
    \setlength{\parindent}{0pt}
    \setlength{\parskip}{6pt plus 2pt minus 1pt}}
}{% if KOMA class
  \KOMAoptions{parskip=half}}
\makeatother
\usepackage{xcolor}
\setlength{\emergencystretch}{3em} % prevent overfull lines
\setcounter{secnumdepth}{-\maxdimen} % remove section numbering
% Make \paragraph and \subparagraph free-standing
\ifx\paragraph\undefined\else
  \let\oldparagraph\paragraph
  \renewcommand{\paragraph}[1]{\oldparagraph{#1}\mbox{}}
\fi
\ifx\subparagraph\undefined\else
  \let\oldsubparagraph\subparagraph
  \renewcommand{\subparagraph}[1]{\oldsubparagraph{#1}\mbox{}}
\fi
\usepackage{color}
\usepackage{fancyvrb}
\newcommand{\VerbBar}{|}
\newcommand{\VERB}{\Verb[commandchars=\\\{\}]}
\DefineVerbatimEnvironment{Highlighting}{Verbatim}{commandchars=\\\{\}}
% Add ',fontsize=\small' for more characters per line
\usepackage{framed}
\definecolor{shadecolor}{RGB}{241,243,245}
\newenvironment{Shaded}{\begin{snugshade}}{\end{snugshade}}
\newcommand{\AlertTok}[1]{\textcolor[rgb]{0.68,0.00,0.00}{#1}}
\newcommand{\AnnotationTok}[1]{\textcolor[rgb]{0.37,0.37,0.37}{#1}}
\newcommand{\AttributeTok}[1]{\textcolor[rgb]{0.40,0.45,0.13}{#1}}
\newcommand{\BaseNTok}[1]{\textcolor[rgb]{0.68,0.00,0.00}{#1}}
\newcommand{\BuiltInTok}[1]{\textcolor[rgb]{0.00,0.23,0.31}{#1}}
\newcommand{\CharTok}[1]{\textcolor[rgb]{0.13,0.47,0.30}{#1}}
\newcommand{\CommentTok}[1]{\textcolor[rgb]{0.37,0.37,0.37}{#1}}
\newcommand{\CommentVarTok}[1]{\textcolor[rgb]{0.37,0.37,0.37}{\textit{#1}}}
\newcommand{\ConstantTok}[1]{\textcolor[rgb]{0.56,0.35,0.01}{#1}}
\newcommand{\ControlFlowTok}[1]{\textcolor[rgb]{0.00,0.23,0.31}{#1}}
\newcommand{\DataTypeTok}[1]{\textcolor[rgb]{0.68,0.00,0.00}{#1}}
\newcommand{\DecValTok}[1]{\textcolor[rgb]{0.68,0.00,0.00}{#1}}
\newcommand{\DocumentationTok}[1]{\textcolor[rgb]{0.37,0.37,0.37}{\textit{#1}}}
\newcommand{\ErrorTok}[1]{\textcolor[rgb]{0.68,0.00,0.00}{#1}}
\newcommand{\ExtensionTok}[1]{\textcolor[rgb]{0.00,0.23,0.31}{#1}}
\newcommand{\FloatTok}[1]{\textcolor[rgb]{0.68,0.00,0.00}{#1}}
\newcommand{\FunctionTok}[1]{\textcolor[rgb]{0.28,0.35,0.67}{#1}}
\newcommand{\ImportTok}[1]{\textcolor[rgb]{0.00,0.46,0.62}{#1}}
\newcommand{\InformationTok}[1]{\textcolor[rgb]{0.37,0.37,0.37}{#1}}
\newcommand{\KeywordTok}[1]{\textcolor[rgb]{0.00,0.23,0.31}{#1}}
\newcommand{\NormalTok}[1]{\textcolor[rgb]{0.00,0.23,0.31}{#1}}
\newcommand{\OperatorTok}[1]{\textcolor[rgb]{0.37,0.37,0.37}{#1}}
\newcommand{\OtherTok}[1]{\textcolor[rgb]{0.00,0.23,0.31}{#1}}
\newcommand{\PreprocessorTok}[1]{\textcolor[rgb]{0.68,0.00,0.00}{#1}}
\newcommand{\RegionMarkerTok}[1]{\textcolor[rgb]{0.00,0.23,0.31}{#1}}
\newcommand{\SpecialCharTok}[1]{\textcolor[rgb]{0.37,0.37,0.37}{#1}}
\newcommand{\SpecialStringTok}[1]{\textcolor[rgb]{0.13,0.47,0.30}{#1}}
\newcommand{\StringTok}[1]{\textcolor[rgb]{0.13,0.47,0.30}{#1}}
\newcommand{\VariableTok}[1]{\textcolor[rgb]{0.07,0.07,0.07}{#1}}
\newcommand{\VerbatimStringTok}[1]{\textcolor[rgb]{0.13,0.47,0.30}{#1}}
\newcommand{\WarningTok}[1]{\textcolor[rgb]{0.37,0.37,0.37}{\textit{#1}}}

\providecommand{\tightlist}{%
  \setlength{\itemsep}{0pt}\setlength{\parskip}{0pt}}\usepackage{longtable,booktabs,array}
\usepackage{calc} % for calculating minipage widths
% Correct order of tables after \paragraph or \subparagraph
\usepackage{etoolbox}
\makeatletter
\patchcmd\longtable{\par}{\if@noskipsec\mbox{}\fi\par}{}{}
\makeatother
% Allow footnotes in longtable head/foot
\IfFileExists{footnotehyper.sty}{\usepackage{footnotehyper}}{\usepackage{footnote}}
\makesavenoteenv{longtable}
\usepackage{graphicx}
\makeatletter
\def\maxwidth{\ifdim\Gin@nat@width>\linewidth\linewidth\else\Gin@nat@width\fi}
\def\maxheight{\ifdim\Gin@nat@height>\textheight\textheight\else\Gin@nat@height\fi}
\makeatother
% Scale images if necessary, so that they will not overflow the page
% margins by default, and it is still possible to overwrite the defaults
% using explicit options in \includegraphics[width, height, ...]{}
\setkeys{Gin}{width=\maxwidth,height=\maxheight,keepaspectratio}
% Set default figure placement to htbp
\makeatletter
\def\fps@figure{htbp}
\makeatother

\makeatletter
\@ifpackageloaded{caption}{}{\usepackage{caption}}
\AtBeginDocument{%
\ifdefined\contentsname
  \renewcommand*\contentsname{Table of contents}
\else
  \newcommand\contentsname{Table of contents}
\fi
\ifdefined\listfigurename
  \renewcommand*\listfigurename{List of Figures}
\else
  \newcommand\listfigurename{List of Figures}
\fi
\ifdefined\listtablename
  \renewcommand*\listtablename{List of Tables}
\else
  \newcommand\listtablename{List of Tables}
\fi
\ifdefined\figurename
  \renewcommand*\figurename{Figure}
\else
  \newcommand\figurename{Figure}
\fi
\ifdefined\tablename
  \renewcommand*\tablename{Table}
\else
  \newcommand\tablename{Table}
\fi
}
\@ifpackageloaded{float}{}{\usepackage{float}}
\floatstyle{ruled}
\@ifundefined{c@chapter}{\newfloat{codelisting}{h}{lop}}{\newfloat{codelisting}{h}{lop}[chapter]}
\floatname{codelisting}{Listing}
\newcommand*\listoflistings{\listof{codelisting}{List of Listings}}
\usepackage{amsthm}
\theoremstyle{plain}
\newtheorem{theorem}{Theorem}[section]
\theoremstyle{remark}
\AtBeginDocument{\renewcommand*{\proofname}{Proof}}
\newtheorem*{remark}{Remark}
\newtheorem*{solution}{Solution}
\newtheorem{refremark}{Remark}[section]
\newtheorem{refsolution}{Solution}[section]
\makeatother
\makeatletter
\makeatother
\makeatletter
\@ifpackageloaded{caption}{}{\usepackage{caption}}
\@ifpackageloaded{subcaption}{}{\usepackage{subcaption}}
\makeatother
\ifLuaTeX
  \usepackage{selnolig}  % disable illegal ligatures
\fi
\usepackage[]{biblatex}
\addbibresource{sample.bib}
\usepackage{bookmark}

\IfFileExists{xurl.sty}{\usepackage{xurl}}{} % add URL line breaks if available
\urlstyle{same} % disable monospaced font for URLs
\hypersetup{
  pdftitle={Author Usage Template for MIT Journals},
  pdfauthor={Author 1; Author 2; Author 3},
  pdfkeywords={keywords1, keywords2, keywords3},
  colorlinks=true,
  linkcolor={blue},
  filecolor={Maroon},
  citecolor={Blue},
  urlcolor={Blue},
  pdfcreator={LaTeX via pandoc}}

\title{Author Usage Template for MIT Journals}

\author{
    Author 1,$^{1\ast}$
    Author 2,$^{1\dag}$
    Author 3$^{2\dag}$
\\
{\small $^{1}$Department of Chemistry, University of Wherever}\\
{\small An Unknown Address, Wherever 0, USA}\\
{\small $^{2}$Another Unknown Address, Palookaville 99999, USA}\\
{\small $^\ast$Correspondence: jsmith@wherever.edu}\\
{\small $^\dag$These authors contributed equally.}\\
}
\begin{document}
\maketitle
\begin{abstract}
This document presents a number of hints about how to set up your paper
in \LaTeX. We provide a template file,
\texttt{imag-ms-template-instr.tex}, that you can use to set up the
\LaTeX~source for your article. An example of the style is the special
\texttt{\{abstract\}} environment used to set up the abstract you see
here.
\end{abstract}

\section{Introduction}\label{introduction}

In this file, we present some tips and sample mark-up to ensure that
your \LaTeX~file has the smoothest possible journey from review
manuscript to published paper. We focus here particularly on issues
related to headings, citations, math, tables, and figures, as those tend
to be the biggest sticking points. Please use the source file for this
document, \texttt{imag-ms-template-instr.tex}, as a template for your
manuscript, cutting and pasting your content into the file at the
appropriate places.

\newpage{}

\subsection{Front matter}\label{front-matter}

Please use the below tags for the article front matter:

\begin{Shaded}
\begin{Highlighting}[]
\FunctionTok{title}\KeywordTok{:}\AttributeTok{ Author Usage Template for MIT Journals}
\FunctionTok{author}\KeywordTok{:}
\AttributeTok{  }\KeywordTok{{-}}\AttributeTok{ }\FunctionTok{name}\KeywordTok{:}\AttributeTok{ Author 1}
\AttributeTok{    }\FunctionTok{email}\KeywordTok{:}\AttributeTok{ jsmith@wherever.edu}
\AttributeTok{    }\FunctionTok{affiliations}\KeywordTok{:}
\AttributeTok{      }\KeywordTok{{-}}\AttributeTok{ }\FunctionTok{id}\KeywordTok{:}\AttributeTok{ wherever}
\AttributeTok{        }\FunctionTok{department}\KeywordTok{:}\AttributeTok{ Department of Chemistry}
\AttributeTok{        }\FunctionTok{name}\KeywordTok{:}\AttributeTok{ University of Wherever}
\AttributeTok{        }\FunctionTok{address}\KeywordTok{:}\AttributeTok{ An Unknown Address}
\AttributeTok{        }\FunctionTok{city}\KeywordTok{:}\AttributeTok{ Wherever}
\AttributeTok{        }\FunctionTok{state}\KeywordTok{:}\AttributeTok{ ST}
\AttributeTok{        }\FunctionTok{country}\KeywordTok{:}\AttributeTok{ USA}
\AttributeTok{        }\FunctionTok{postal{-}code}\KeywordTok{:}\AttributeTok{ }\DecValTok{00000}
\AttributeTok{    }\FunctionTok{attributes}\KeywordTok{:}
\AttributeTok{      }\FunctionTok{corresponding}\KeywordTok{:}\AttributeTok{ }\CharTok{true}
\AttributeTok{  }\KeywordTok{{-}}\AttributeTok{ }\FunctionTok{name}\KeywordTok{:}\AttributeTok{ Author 2}
\AttributeTok{    }\FunctionTok{affiliations}\KeywordTok{:}
\AttributeTok{      }\KeywordTok{{-}}\AttributeTok{ }\FunctionTok{ref}\KeywordTok{:}\AttributeTok{ wherever}
\AttributeTok{  }\KeywordTok{{-}}\AttributeTok{ }\FunctionTok{name}\KeywordTok{:}\AttributeTok{ Author 3}
\AttributeTok{    }\FunctionTok{affiliations}\KeywordTok{:}
\AttributeTok{      }\KeywordTok{{-}}\AttributeTok{ }\FunctionTok{address}\KeywordTok{:}\AttributeTok{ Another Unknown Address}
\AttributeTok{        }\FunctionTok{city}\KeywordTok{:}\AttributeTok{ Palookaville}
\AttributeTok{        }\FunctionTok{state}\KeywordTok{:}\AttributeTok{ ST}
\AttributeTok{        }\FunctionTok{postal{-}code}\KeywordTok{:}\AttributeTok{ }\DecValTok{99999}
\AttributeTok{        }\FunctionTok{country}\KeywordTok{:}\AttributeTok{ USA}
\FunctionTok{abstract}\KeywordTok{: }\CharTok{|}
\NormalTok{  This document presents a number of hints about how to set up your}
\NormalTok{  paper in \textbackslash{}LaTeX.  We provide a template file,}
\NormalTok{  \textasciigrave{}imag{-}ms{-}template{-}instr.tex\textasciigrave{}, that you can use to set up the}
\NormalTok{  \textbackslash{}LaTeX\textbackslash{} source for your article.  An example of the style is the}
\NormalTok{  special \textasciigrave{}\{abstract\}\textasciigrave{} environment used to set up the}
\NormalTok{  abstract you see here.}
\FunctionTok{keywords}\KeywordTok{:}\AttributeTok{ }\KeywordTok{[}\AttributeTok{keywords1}\KeywordTok{,}\AttributeTok{ keywords2}\KeywordTok{,}\AttributeTok{ keywords3}\KeywordTok{]}
\FunctionTok{author\_contributions}\KeywordTok{: }\CharTok{|}
\NormalTok{  Please provide details of author contributions here.}
\FunctionTok{conflict\_of\_interest}\KeywordTok{: }\CharTok{|}
\NormalTok{  Please declare any competing interests here.}
\FunctionTok{acknowledgements}\KeywordTok{: }\CharTok{|}
\NormalTok{  Please include your acknowledgments here, set in a single paragraph. Please do not}
\NormalTok{  include any acknowledgments in the Supporting Information, or anywhere else in the}
\NormalTok{  manuscript.}
\FunctionTok{code\_availability}\KeywordTok{: }\CharTok{|}
\NormalTok{  Data and Code Availability text (mandatory unless there is no data or code used).}
\FunctionTok{funding}\KeywordTok{: }\CharTok{|}
\NormalTok{  Funding text (optional).}
\end{Highlighting}
\end{Shaded}

\subsection{Headings}\label{headings}

Use the standard tags
\texttt{\textbackslash{}section,\ \textbackslash{}subsection,\ \textbackslash{}subsubsection,\ \textbackslash{}paragraph},
and \texttt{\textbackslash{}subparagraph} for the Headings
\texttt{H1,\ H2,\ H3,\ H4}, and \texttt{H5}, respectively.

\subsection{Handling Math, Tables, and
Figures}\label{handling-math-tables-and-figures}

We suggest using the \texttt{mathtools.sty} file to get various display
math styles. A few of the codes are given below for easy reference:

\bigskip

\noindent

\begin{tabular}{@{}ll}
\toprule
\verb!equation!\\
\verb!align!\\
\verb!\[...\]! or \verb!equation*!\\
\verb!gather!\\
Various types of matrices, e.g., \verb!pmatrix!, \verb!bmatrix!,
\verb!vmatrix!, \verb!smallmatrix!,\\
\verb!alignat!, etc.\\
\bottomrule
\end{tabular}

\subsection{Tables}\label{tables}

We suggest using the \verb!threeparttable.sty! file to format the tables
and their notes properly. Examples are given below:

\begin{verbatim}
\begin{table}
\begin{threeparttable}
\caption{Time of the Transition Between Phase 1 and Phase 2\tnote{$a$}
\label{tab:label}}
\begin{tabular}{@{}ll}
\toprule
 Run  & Time (min)  \\
\midrule
  \textit{l}1  & 260   \\
  \textit{l}2  & 300   \\
  \textit{l}3  & 340   \\
  \textit{h}1  & 270   \\
  \textit{h}2  & 250   \\
  \textit{h}3  & 380   \\
  \textit{r}1  & 370   \\
  \textit{r}2  & 390   \\
\bottomrule
\end{tabular}
\begin{tablenotes}[flushleft]\footnotesize
\item[${a}$]Table note text here.
\end{tablenotes}
\end{threeparttable}
\end{table}
\end{verbatim}

\subsubsection*{Output}\label{output}

\begin{table}[h!]
\begin{threeparttable}
\caption{Time of the Transition Between Phase 1 and Phase 2\tnote{$a$}
\label{tab:label}}
\setlength{\tabcolsep}{45pt}%
\begin{tabular}{@{}ll}
\toprule
 Run  & Time (min)  \\
\midrule
  \textit{l}1  & 260   \\
  \textit{l}2  & 300   \\
  \textit{l}3  & 340   \\
  \textit{h}1  & 270   \\
  \textit{h}2  & 250   \\
  \textit{h}3  & 380   \\
  \textit{r}1  & 370   \\
  \textit{r}2  & 390   \\
\bottomrule
\end{tabular}
\begin{tablenotes}[flushleft]\footnotesize
\item[${a}$]Table note text here.
\end{tablenotes}
\end{threeparttable}
\end{table}

\subsection*{Spanning rules}\label{spanning-rules}

Use \verb!\cmidrule! to obtain spanning of rules from column to column.
Usage is

\begin{verbatim}
\cmidrule{fromcolumn-tocolumn}, e.g., \cmidrule{2-3}.
\end{verbatim}

\subsection{Figures}\label{figures}

Figure callouts within the text should be in the form of
\LaTeX~references; for example, \verb+\ref{fig1}+.

For inclusion of figures (e.g., Figure~\ref{fig-example_figure}), please
use code such as:

\begin{verbatim}
![Example caption text](figure.pdf){#fig-example_figure}
\end{verbatim}

\begin{figure}

\centering{

\includegraphics{figure.pdf}

}

\caption{\label{fig-example_figure}Example caption text}

\end{figure}%

Please use
\texttt{\textbackslash{}begin\{sidewaystable\}...\textbackslash{}end\{sidewaystable\}}
and\\
\texttt{\textbackslash{}begin\{sidwaysfigure\}...\textbackslash{}end\{sidewaysfigure\}}
to get rotating figures/tables.

\section{Algorithms}\label{algorithms}

For \verb@Algorithms@, please use the standard \LaTeX~supporting file
\verb!algorithm2e.sty!; the format and the output are given below:

\begin{verbatim}
\begin{algorithm}[h!]
\SetAlgoLined
\SetKwFunction{IL}{InitializeDistance}
\SetKwFunction{PL}{PropagateInsertion}
\SetKwFunction{MIN}{Min}
\SetKwFunction{MX}{Max}
\SetKwFunction{TOP}{Top}
\SetKwFunction{Push}{Push}
\SetKwFunction{Pop}{Pop}
\SetKwFunction{Append}{Append}
\SetKwData{Queue}{Queue}
\KwResult{The length of shortest path from $s$ to $t$}
 $PreviousLayer=[s]$\;
 $s.distance = 0$\;
 \For(\tcc*[f]{Do the computation layer by layer}){i = 1 \KwTo m}{
   $CurrentLayer = [(i,v_1),(i,v_{2}),\ldots, (i,v_{n}), (i,k)]$\;
   $x.distance = \infty \ \forall  x \in CurrentLayer$\;
   \IL{PreviousLayer,CurrentLayer}\;
   \PL{CurrentLayer}\;
   $PreviousLayer = CurrentLayer$\;
 }
 \KwRet{\MIN{PreviousLayer.distance}}\;
 \caption{Algorithm for sequence to graph alignment}
 \label{algo:linear}
\end{algorithm}
\end{verbatim}

\subsection*{Output}\label{output-1}

\begin{algorithm}[h!]
\SetAlgoLined
\SetKwFunction{IL}{InitializeDistance}
\SetKwFunction{PL}{PropagateInsertion}
\SetKwFunction{MIN}{Min}
\SetKwFunction{MX}{Max}
\SetKwFunction{TOP}{Top}
\SetKwFunction{Push}{Push}
\SetKwFunction{Pop}{Pop}
\SetKwFunction{Append}{Append}
\SetKwData{Queue}{Queue}
\KwResult{The length of shortest path from $s$ to $t$}
 $PreviousLayer=[s]$\;
 $s.distance = 0$\;
 \For(\tcc*[f]{Do the computation layer by layer}){i = 1 \KwTo m}{
   $CurrentLayer = [(i,v_1),(i,v_{2}),\ldots, (i,v_{n}), (i,k)]$\;
   $x.distance = \infty \ \forall  x \in CurrentLayer$\;
   \IL{PreviousLayer,CurrentLayer}\;
   \PL{CurrentLayer}\;
   $PreviousLayer = CurrentLayer$\;
 }
 \KwRet{\MIN{PreviousLayer.distance}}\;
 \caption{Algorithm for sequence to graph alignment}
 \label{algo:linear}
\end{algorithm}

\section{Lists}\label{lists}

Please use the standard tags for Numbered lists and Bulleted lists; for
example,

\subsection*{Numbered lists}\label{numbered-lists}

\begin{verbatim}
1. Text for first-level numbered lists text text text text
   Text for first-level numbered lists text text text text: 
  a. Text for second level numbered lists text text text text Text for second level numbered lists text text text text 
  b. Text text text text Text for second level numbered lists text text text text 
2. Text text text text
   Text for first-level numbered lists text text text text 
\end{verbatim}

\subsection*{Output}\label{output-2}

\begin{enumerate}
\def\labelenumi{\arabic{enumi}.}
\tightlist
\item
  Text for first-level numbered lists text text text text Text for
  first-level numbered lists text text text text:

  \begin{enumerate}
  \def\labelenumii{\alph{enumii}.}
  \tightlist
  \item
    Text for second level numbered lists text text text text Text for
    second level numbered lists text text text text
  \item
    Text text text text Text for second level numbered lists text text
    text text
  \end{enumerate}
\item
  Text text text text Text for first-level numbered lists text text text
  text
\end{enumerate}

\subsection*{Bulleted lists}\label{bulleted-lists}

\begin{verbatim}
* Text for first-level bulleted lists text text text text
  Text for first-level bulleted lists text text text text 
    + text for second level bulleted lists text text text text
      Text for second level bulleted lists text text text text 
    + text text text text Text for second level bulleted lists
      text text text text 
* Text text text text Text for first-level bulleted lists
  text text text text 
\end{verbatim}

\subsection*{Output}\label{output-3}

\begin{itemize}
\tightlist
\item
  Text for first-level bulleted lists text text text text Text for
  first-level bulleted lists text text text text

  \begin{itemize}
  \tightlist
  \item
    text for second level bulleted lists text text text text Text for
    second level bulleted lists text text text text
  \item
    text text text text Text for second level bulleted lists text text
    text text
  \end{itemize}
\item
  Text text text text Text for first-level bulleted lists text text text
  text
\end{itemize}

\subsection*{Extract/Quote}\label{extractquote}

Use the standard tag \verb!\begin{quote}...\end{quote}! for quoted text;
for example,

\begin{verbatim}
> Text for quoted text text text text text text text text text text text
> text text text text text text text text text text text text text text 
\end{verbatim}

\subsection*{Output}\label{output-4}

\begin{quote}
Text for quoted text text text text text text text text text text text
text text text text text text text text text text text text text text
\end{quote}

\section{Footnote}\label{footnote}

Use a Quarto footnote\footnote{This is a footnote} to get footnotes at
the bottom of the page.

\section{Special Fonts}\label{special-fonts}

Use the standard \LaTeX~tags \verb!\mathcal!, \verb!\mathscr!, and
\verb!\mathbb! to get characters in special fonts such as
\(\mathcal{A}, \mathscr{A}\), and \(\mathbb{A}\), respectively.

\section{Enunciation or Math Heads}\label{enunciation-or-math-heads}

Generally \verb!theorem!, \verb!lemma!, etc., are called Enunciation or
Math heads. In this template, we define some standard enunciations
(\verb!theorem!, \verb!lemma!, \verb!corollary!).

\subsection*{Sample Input/Output}\label{sample-inputoutput}

\subsubsection*{Input}\label{input}

\begin{verbatim}
::: {#thm-line}
The equation of any straight line, called a linear equation, can be written as:

$$
y = mx + b
$$
:::
\end{verbatim}

\subsubsection*{Output}\label{output-5}

\begin{theorem}[]\protect\hypertarget{thm-line}{}\label{thm-line}

The equation of any straight line, called a linear equation, can be
written as:

\[
y = mx + b
\]

\end{theorem}

\subsection{Define Own Math
Heads/Enunciation}\label{define-own-math-headsenunciation}

You are allowed to define your own enunciations; the format is given
below:

\begin{verbatim}
\newtheorem{short name of the head}{Head to Display}
\end{verbatim}

\subsubsection*{Example}\label{example}

If you need to define a group of text under the head ``Proposition,'\,'
then you have to define it as

\begin{verbatim}
\newtheorem{proposition}{Proposition}
\end{verbatim}

\newtheorem{proposition}{Proposition}
\begin{proposition}
This is a test for math head ``Proposition'' text text text text
\end{proposition}

\subsection{Unnumbered Math
Heads/Enunciation}\label{unnumbered-math-headsenunciation}

Just introduce \verb!*!, which makes the numbered math head text into an
unnumbered math head; for example,

\begin{verbatim}
\begin{theorem*}
This is a test for unnumbered math head ``Theorem'' text text text text
\end{theorem*}
\end{verbatim}

\begin{theorem*}
This is a test for unnumbered math head ``Theorem'' text text text text
\end{theorem*}

\section{Bibliography/References with APA
Style}\label{bibliographyreferences-with-apa-style}

As per MIT standards, we fixed the Reference style \verb!APA! in the
template with the combination of the supporting file \verb!biblatex! and
\verb!natbib! options, which help to achieve various types of
bibliography cross links. Those details are given below:

\subsection{Formatting Citations}\label{formatting-citations}

\begin{longtable}[]{@{}ll@{}}
\toprule\noalign{}
Type & Results \\
\midrule\noalign{}
\endhead
\bottomrule\noalign{}
\endlastfoot
\texttt{@ref2} & Goossens et al.~(1993) \\
\texttt{@ref2\ {[}chap.\ 2{]}} & Goossens et al.~(1993, chap.~2) \\
\texttt{{[}@ref2{]}} & (Goossens et al., 1993) \\
\texttt{{[}@ref2,\ chap.\ 2{]}} & (Goossens et al., 1993, chap.~2) \\
\texttt{{[}see\ @ref2{]}} & (see Goossens et al., 1993) \\
\texttt{{[}see\ @ref2,\ chap.\ 2{]}} & (see Goossens et al., 1993,
chap.~2) \\
\end{longtable}

\bigskip

\textbf{Note}: Please use \textbf{biber} (biber.exe in Windows) to get
better output for References.

\nocite{*}

\subsection{Example Citations}\label{example-citations}

See \textcite{Einstein1905} and \autocite{Goossens1993,Knuth1986}. Also
see \textcite{Chen2023}.

\section{Note to User}\label{note-to-user}

We have already included almost all the required \texttt{.sty} files in
the \LaTeX~template \verb!imag-ms-template.cls!; hence, there is no need
to call those in your \texttt{.tex} application files.

\subsection*{General Notes}\label{general-notes}

\noindent This template will work in most recent \TeX~distributions
(e.g., MiKTeX, TeXLive) with any type of \TeX~engines, such as \LaTeX,
PDF\LaTeX, Xe\LaTeX, and Lua\LaTeX, as well as in all types of OS, such
as MS-Windows, Mac OS X, and Linux. It will also work well in Overleaf.

\section*{Data and Code Availability}

Data and Code Availability text (mandatory unless there is no data or
code used).

\section*{Author Contributions}

Please provide details of author contributions here.

\section*{Funding}

Funding text (optional).

\section*{Declaration of Competing Interests}

Please declare any competing interests here.

\section*{Acknowledgements}

Please include your acknowledgments here, set in a single paragraph.
Please do not include any acknowledgments in the Supporting Information,
or anywhere else in the manuscript.

\printbibliography


\end{document}
